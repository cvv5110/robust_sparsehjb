\documentclass{article}
\usepackage{graphicx}
\usepackage[margin=1in]{geometry}
\usepackage{amssymb}
\usepackage{amsmath}

\title{SparseHJB With Robust Output Feedback Regulation}
\author{Carlos A. Vargas Venegas}
\date{January 2026}

\newcommand{\vs}{\mathbf{s}}
\newcommand{\vx}{\mathbf{x}}
\newcommand{\vu}{\mathbf{u}}
\newcommand{\vz}{\mathbf{z}}
\newcommand{\cS}{\mathcal{S}}
\newcommand{\cU}{\mathcal{U}}
\newcommand{\cT}{\mathcal{T}}
\newcommand{\cP}{\mathcal{P}}
\newcommand{\vp}{\mathbf{p}}
\newcommand{\vy}{\mathbf{y}}
\newcommand{\vv}{\mathbf{v}}
\newcommand{\cZ}{\mathcal{Z}}
\newcommand{\vf}{\mathbf{f}}
\newcommand{\vh}{\mathbf{h}}
\newcommand{\vr}{\mathbf{r}}
\newcommand{\ve}{\mathbf{e}}
\newcommand{\vzero}{\boldsymbol{0}}
\newcommand{\cV}{\mathcal{V}}
\newcommand{\cR}{\mathcal{R}}
\newcommand{\vsigma}{\boldsymbol{\sigma}}
\newcommand{\vgamma}{\boldsymbol{\gamma}}
\newcommand{\vPsi}{\boldsymbol{\Psi}}
\newcommand{\vlambda}{\boldsymbol{\lambda}}
\newcommand{\cJ}{\mathcal{J}}
\newcommand{\cL}{\mathcal{L}}

\begin{document}

\maketitle

\section{Problem Statement}

Define the following quantities,
\begin{itemize}
    \item States: $\vs\in\mathcal{S}\subset\mathbb{R}^{n_s}$,
    \item Outputs: $\vz\in\cZ\subset\mathbb{R}^{n_z}$,
    \item Controls: $\vu\in\mathcal{U}\subset\mathbb{R}^{n_u}$,
    \item Parameters: $\vp\in\cP\subset\mathbb{R}^{n_p}$,
    \item Reference signal: $\vr(t)\in\cR\subset\mathbb{R}^{n_r}$,
    \item Time: $t\in\left[t_0,t_f\right]\subset\mathcal{T}$
\end{itemize}
The states are governed by the following model of the plant,
\begin{subequations}
    \begin{align}
        \dot{\vs} &= \vf\left(\vs,\vu;\vp\right)\\
        \vz &= \vh\left(\vs;\vp\right)
    \end{align}
\end{subequations}

The basic goal of stabilization is to design the control input so that the controlled output $\vz(t)$ tracks a reference signal $\vr(t)$; that is,
\begin{equation}
    \ve(t) = \vz(t) - \vr(t) \approx \vzero,\quad \forall t\geq t_0
\end{equation}
where $t_0$ is the time at which control starts. Since the initial value of $\vz(t)$ depends on the initial state $\vs(t_0)$, meeting this requirement for all $t\geq t_0$ would require either presetting $\vs(t_0)$ or presetting the initial value of the reference signal by assuming knowledge of $\vs(t_0)$, which is not feasible in many practical applications. Therefore, an asymptotic output tracking goal is often sought, where the tracking error $\ve(t)$ approaches zero as $t\rightarrow\infty$; that is,
\begin{equation}
    \ve(t)\rightarrow\vzero,\:\text{as}\:t\rightarrow\infty
\end{equation}
In the case of constant exogenous signals, where the goal is to asymptotically regulate $\vz(t)$ to a \emph{set point} $\vr$, asymptotic regulation and disturbance rejection (e.g., with respect to $\vp$) can be achieved by including \emph{integral action} in the feedback controller. This is the only way to achieve asymptotic regulation in the presence of parametric uncertainties, which explains the popularity of proportional-integral (PI) and proportional-integral-derivative (PID) controllers in industrial applications.

\section{Integral Control}

In this section, an integral control approach is presented that ensures asymptotic regulation under all parameter perturbations that do not destroy the stability of the closed-loop system. Let $\vr$ be a constant reference that is available online, and set,
\begin{equation}
    \vv = \left[\vr,\vp\right]^\top\in\cV=\cR\times\cP
\end{equation}
It is desired to design a feedback controller such that,
\begin{equation}
    \vz(t)\rightarrow\vr\:\text{as}\:\rightarrow\infty
\end{equation}
The regulation task is achieved by stabilizing the system at an equilibrium point where $\vz(t)=\vr$. It is assumed that for each $\vv\in\cV$, there is a unique pair $\left(\vs_{f},\vu_{f}\right)$ that depends continuously on $\vv$ and satisfies the equations,
\begin{subequations}
    \begin{align}
        \vzero &= \vf\left(\vs_{f},\vu_{f};\vp\right)\\
        \vr &= \vh(\vs_{f};\vp)
    \end{align}
\end{subequations}
so that $\vs_f$ is the desired equilibrium point and $\vu_f$ is the steady-state control that is needed to maintain equilibrium at $\vs_f$.

To introduce integral action, the regulation error is integrated,
\begin{equation}
    \dot{\sigma} = \ve(t) = \vz(t) - \vr
\end{equation}
and the governing dynamics are augmented as,
\begin{subequations}
\label{eqn_augmented_dynamics}
    \begin{align}
        \dot{\vs} &= \vf\left(\vs,\vu;\vp\right)\\
        \dot{\vsigma} &= \vh(\vs;\vp) - \vr
    \end{align}
\end{subequations}
It is clear that integrating $\ve(t)$ requires both $\vz(t)$ and $\vr$ to be available online. The control task now is to design a stabilizing feedback controller that stabilizes the augmented state model at an equilibrium point $\left(\vs_f,\vsigma_f\right)$ where $\vsigma_f$ produces the desired $\vu_f$.

The integral form of the controller comprises two components: the \emph{integrator} and the \emph{stabilizer}. The integrator is sometimes called the internal model, since it duplicates the model of the equation $\dot{\vv}=\vzero$, which generates the exogenous constant signal $\vv$. The stabilizer depends on the measured signal. For example, in the case of state feedback; that is, when $\vz(t)=\vs$, the stabilizing controller takes the form,
\begin{equation}
    \vu(t) = \vgamma\left(\vs,\vsigma,\ve\right)
\end{equation}
where $\vgamma$ is designed such that there is a unique $\vsigma_f$ that satisfies the equation,
\begin{equation}
    \vgamma\left(\vs_f,\vsigma_f,\vzero\right) = \vu^*(t_f)
\end{equation}
Therefore, the closed-loop system,
\begin{subequations}
    \begin{align}
        \dot{\vs} &= \vf\left(\vs,\vgamma(\vs,\vsigma,\ve);\vp\right)\\
        \dot{\vsigma} &= \vh(\vs;\vp) - \vr\\
        \ve &= \vz(t) - \vr
    \end{align}
\end{subequations}
has an asymptotically stable equilibrium point at $\left(\vs_f,\vsigma_f\right)$. At the equilibrium point, $\vz = \vr$, irrespective of the parameter $\vp$. Hence, asymptotic regulation is achieved for all initial states in the region of attraction of $\left(\vs_f,\vsigma_f\right)$.

The fact that the integral controller is robust to all parameter perturbations that do not destroy the stability of the closed-loop system can be intuitively explained as follows: The feedback controller creates an asymptotically-stable equilibrium point. At this point, all signals must be constant. For the integrator $\dot{\vsigma} = \ve$ to have a constant output $\vsigma$, its input $\ve(t)$ must be zero. Therefore, the inclusion of the integrator forces the regulation error to be zero at equilibrium. Parameter perturbations will change the equilibrium point, but the condition $\ve=0$ at equilibrium is maintained. Thus, as long as the perturbed equilibrium point remains asymptotically stable, regulation will be achieved.

\section{Lyapunov Re-design}

A Lyapunov function $\cL$ for

\section{Value Function Learning}

Consider the following optimal control problem (OCP),
\begin{subequations}
    \begin{align}
        \min_{\vu}\:\mathcal{J} &= \varphi\left(\vs(t_f),t_f\right) + \int_{t_0}^{t_f}\ell\left(\vs,\vu,\vp\right)dt\\
        \text{s.t.}\: \dot{\vs} &= \vf\left(\vs,\vu;\vp\right)\\
        \vzero &= \vPsi_f\left(\vs(t_f),t_f;\vp\right)\\
        \vzero &= \vPsi_0\left(\vs(t_0),t_0;\vp\right)
    \end{align}
\end{subequations}
Solutions to this problem include the optimal trajectories,
\begin{equation}
    d_l = \left\{\vs^*_l(t),\vu_l^*(t),\vlambda^*_l(t),\vp_l\right\}_{l=1}^{n_p},\:\forall\:t\in\cT
\end{equation}
which extremize the original objective $\cJ$ while satisfying the constraints exactly. The OCP solutions are effectively regulating the non-linear dynamics $\vf$ to a point such that $\vPsi_f = \vzero$.

The value function for the original problem is,
\begin{equation}
    V(\vs,\vp,t) = \min_{\vu}\left\{\varphi\left(\vs(t_f),t_f\right) + \int_{t_0}^{t_f}\ell(\vs,\vu,\vp)dt\right\}
\end{equation}
however, including the regulation dynamics by considering Eq.~\ref{eqn_augmented_dynamics}, the value function includes the extended state space $\vy = \left[\vs,\vsigma\right]^\top$

\end{document}
